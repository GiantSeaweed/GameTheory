%!TEX program = xelatex
\documentclass[a4papers]{ctexart}
%数学符号
\usepackage{amssymb}
\usepackage{amsmath}
%表格
\usepackage{graphicx,floatrow}
\usepackage{array}
\usepackage{booktabs}
\usepackage{makecell}
%页边距
\usepackage{geometry}
\geometry{left=2cm,right=2cm,top=1cm,bottom=2cm}

%首行缩进两字符 利用\indent \noindent进行控制
\usepackage{indentfirst}
\setlength{\parindent}{2em}

\setromanfont{Songti SC}
%\setromanfont{Heiti SC}

\usepackage{pstricks,pst-plot}


\title{Game Theory \\Homework 1}
\author{161220039 冯诗伟 }
\date{}
\begin{document}
\maketitle
\section*{Exercise a}


\section*{Exercise b}


\section*{Exercise c}


\section*{Exercise d}
\noindent Prove that in n-Cournet case, the Nash Equilibria is given by
\[ \Big\{ \Big(\dfrac{a-c}{(n+1)b},\cdots,\dfrac{a-c}{(n+1)b}  \Big) \Big\} \]
Proof:\\ Assmue that $(q_1^*,q_2^*,\cdots,q_n^*)$ is a Nash equilibrium.\\
First, prove that $q_i^*$ > 0 by contradiction. Suppose that $q_i^* = 0,$ for $i=1,2,\cdots,n$.
Then we get $u(q_1^*,q_2^*,\cdots,q_n^*)=(a-b(q_1^*+q_2^*+\cdots+q_n^*)-c)q_i^* =0,$ for $i=1,2,\cdots,n$.
So, $q_i^*$ can not equal 0 at the same time.\\
Without the loss of generality, assume that $q_1^*=0$, while $q_i > 0$, for $i=2,3\cdots,n$.\\
From
\[\dfrac{\partial u_i(q_1^*,q_2^*,\cdots,q_n^*)}{\partial q_i} = 0  & {i=2,3,\codts,n}\\\]
We get \[q_i = \dfrac{\]

\section*{Exercise e}



    
\end{document}